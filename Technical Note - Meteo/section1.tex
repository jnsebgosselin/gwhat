\documentclass[ARTICLETHERMIC.tex]{subfiles}

\begin{document}

\subsection{Background}

The monitoring of groundwater levels on a daily, and even hourly basis has become a popular practice for groundwater characterization projects since the commercialization in the 1990s of affordable submersible pressure transducers combined with electronic data loggers (Freeman et al., 2004). In Canada for instance, the monitoring network that is currently being deployed by the Quebec government should eventually monitor groundwater levels at a 6 hours interval in approximately 240 observation wells distributed throughout the occupied territory of the province of Quebec (Government of Quebec, 2014). In Saskatchewan, the Water Security Agency is monitoring the groundwater levels (on a daily basis?) with automatic water level recorders in 70 observation wells (Water Security Agency, 2014). Within the United States, the US Geological Survey maintains the largest database on real-time and historic groundwater levels (more details to add) (Healy, 2010).
With the increase in the frequency of measurement of groundwater levels emerged the need to improve and develop new methods to properly exploit these data. Rasmussen and Crawford presented a method to determined the barometric response function of a well, that characterizes the water-level response over time to a step change in barometric pressure, from groundwater level and barometric time series using regression convolution. The BRF could be used as as a diagnostic information about whether the aquifer is confined or not, the presence of boreholes storage or skin effects, and the air diffusivity coefficient within the unsaturated zone. This approach was since then implemented in user friendly computer programs by Toll and Ramsussen (2007) and by Bohling et al. (2011). Posavec et al. (2006, 2010) presented a spreadsheet application to automatically construct a master recession curve (MRC), using the adapted matching strip method, that characterized the recession of water level in period where groundwater recharge to the water table is negligible that can be used for the estimation of groundwater recession in unconfined conditions. Similarly, Heppner and Nimmo (2005) developed a computer program for predicting recharge with a master recession curve. These approach are a good alternative to the more conventional water table fluctuation method is still a widely used method to estimate groundwater recharge in unconfined condition (Healy and Cook, 2002). Finally, building on Baalousha et al. (2005) work, Lefebvre et al. (2011) presented a method to estimate groundwater recharge using weather data time series and the synthetic hydrograph approach for a sandy aquifer deltaic machin in.


Even though significant progresses have been made in the methods and tools available for the valorization and interpretation of groundwater level time-series,  L'application de ces tâches représente un investissement en temps parfois importants pour la manipulation des données et des outputs de chacun des outils. La simple mise en page des hydrogrammes et des données météo, qui est une pratique courante en hydrogéologie, peut être une tâche très longue, et particulièrement si le projet est gros. valorization and interpretation of these data can often prove to be difficult and very time-consuming. However, the task of preparing, plotting and interpreting the data can often prove to be difficult and time consuming, especially in the case of a project with many wells distributed throughout a large study area. The interpretation of groundwater level time series can also provide meaningful insight on the aquifer confinement level and hydrodynamic. 
Est présenté ici un logiciel qui permet de centraliser plusieurs étape du workflow de l'interprétation des données météos, incluant entre autre le téléchargement et le formattage des données météo canadienne, le remplissage des données manquantes, le traçage des hydrogrammes de puits, l'estimation de la courbe maîtresse de récession et l'estimation de la recharge par la méthode MRC et des hydrogrammes de puits synthétique. L'estimation de la recharge une version amélirée de la méthode présenté par Lefebvre. A user friendly, menu-driven, and interactive computer program for rapid and automatic completion of daily climatological series has been developed. Missing data for a given weather station are estimated using a multiple linear regression model, generated using data from nearby stations. For daily precipitation, it is possible to activate an option that forces the algorithm to preserve the probability distribution of data. This is an advantage over conventional approaches that tend to overestimate the number of wet days and underestimate the high intensity precipitation events. The software also allows downloading and automatic formatting of raw data available on the Environment Canada website. The software is demonstrated for two weather station located in Monteregie Est region, southern Quebec. Cross-validation was used to check the method and to define the optimal parameters to minimize the error in estimating missing daily precipitation. We also believe that. 

First of all, almost all the instrumental climatological time series are affected by a percentage of missing values. Estimation of these missing values can prove to be quite a tedious and complicated task. The situation becomes particularly complicated when dealing with precipitation, because of its large space and time variability (Simolo et al., 2010). Finally, estimation of groundwater recharge can be very daunting for long time-series and can be difficult for water level that fluctuates very rapidly due to shallow water table. Though originally developped to ease the estimation of groundwater recharge estimation, meteorological data are useful in a lot of applications in the fields of hydrogeology, hydrology and agronomy. Missing climatological data are serious hindrance to the use of climate-dependent models that are nowadays commonly used in several domain, including hydrology, hydrogeology and agronomy. We expect that the sofware presented herein could provide a usefull tool to many common problem encountered in many fields of Earth Science.

\end{document}